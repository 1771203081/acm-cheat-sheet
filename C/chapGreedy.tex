\chapter{贪心法}
我们前面见过的一些算法,比如单源最短路径、最小生成树等都属于贪心法(greedy algorithm)。

如果一个问题具有以下两个要素:
\begindot
\item 最优子结构(optimal substructure)
\item 贪心选择性质(greedy-choice property)
\myenddot
则可以用贪心法求最优解。


\section{最优装载} %%%%%%%%%%%%%%%%%%%%%%%%%%%%%%


\section{哈弗曼编码} %%%%%%%%%%%%%%%%%%%%%%%%%%%%%%
\subsubsection{描述}
给定一个英文字符串,使用0和1对其进行编码,求最优前缀编码,使其所需要的比特数最少。

\subsubsection{分析}
题目很长,不过就是哈弗曼编码。

\subsubsection{代码}
\begin{Codex}[label=poj1521_entropy.cpp]
// 本题考查哈弗曼编码,但只需要统计哈弗曼编码后的总码长即可,
// 没必要建哈弗曼树得出哈弗曼编码
#include <stdio.h>
#include <string.h>
#include <queue>
#include <functional>

const int LINE_MAX = 256; // 一行最大字符数
const int MAX_ASCII = 128; // ASCII码最大值

int main_entropy() {
    char    s[LINE_MAX];
    int     count[MAX_ASCII] = {0}; // count[i]记录ASCII码为i的字符的出现次数
    int     sum;
    // 小根堆,队列头为最小元素
    priority_queue<int, vector<int>, greater<int> >    pq;

    while (scanf("%s", s) > 0) {
        sum = 0; // 清零
        const int len = strlen(s);

        if (strcmp(s,"END") == 0) {
            break;
        }

        for (int i = 0; i < len; i++) {
            count[s[i]]++;
        }

        for (int i = 0;i < MAX_ASCII; i++) {
            if (count[i] > 0) {
                pq.push(count[i]);
                count[i] = 0;
            }
        }
        while (pq.size() > 1) {
            const int a = pq.top(); pq.pop();
            const int b = pq.top(); pq.pop();
            sum += a + b; 
            pq.push(a + b);
        }
        if (sum == 0) {
            sum = len; // 此时pq中只有一个元素
        }
        
        while (!pq.empty()) { // clear
            pq.pop();
        }
        // 注意精度设置
        printf("%d %d %.1f\n", 8 * len, sum, ((double)8 * len) / sum);
    }
    return 0;
}
\end{Codex}

\subsubsection{相关的题目}
与本题相同的题目:
\begindot
\item POJ 1521 Entropy, \myurl{http://poj.org/problem?id=1521}
\myenddot

与本题相似的题目:
\begindot
\item  POJ 3253 Fence Repair, \myurl{http://poj.org/problem?id=3253}
\item 《算法竞赛入门经典》\footnote{刘汝佳,算法竞赛入门经典,清华大学出版社,2009} 第155页例题8-5
\item  《Introduction to Algorithms》\footnote{CLRS,Introduction to Algorithms(3rd Edition), 2009} 第16.3节
\item 《算法设计与分析(第3版)》\footnote{王晓东,计算机算法设计与分析(第3版), 2007} 第109页4.4节
\myenddot


\section{部分背包问题} %%%%%%%%%%%%%%%%%%%%%%%%%%%%%%
