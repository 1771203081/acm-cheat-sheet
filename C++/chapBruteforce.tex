\chapter{暴力枚举法}

\section{算法思想} %%%%%%%%%%%%%%%%%%%%%%%%%%%%%%
生成--测试法。

\section{简单枚举} %%%%%%%%%%%%%%%%%%%%%%%%%%%%%%

\subsection{分数拆分}
输入正整数$k$,找到所有的正整数 $x \geq y$,使得 $\dfrac{1}{k}=\dfrac{1}{x}+\dfrac{1}{y}$。

\textbf{样例输入}\\
2 \\
12 \\

\textbf{样例输出} \\
2 \\
1/2=1/6+1/3 \\
1/2=1/4+1/4 \\
8
1/12=1/156+1/13 \\
1/12=1/84+1/14 \\
1/12=1/60+1/15 \\
1/12=1/48+1/16 \\
1/12=1/36+1/18 \\
1/12=1/30+1/20 \\
1/12=1/28+1/21 \\
1/12=1/24+1/24 \\

\subsubsection{分析}
既然说找出所有的$x,y$,枚举对象自然就是他们了。可问题在于:枚举范围如何?
从$\dfrac{1}{12}=\dfrac{1}{156}+\dfrac{1}{13}$,可以看出,$x$可以比$y$大很多。
难道要无休止地枚举下去?当然不是。由于$x \geq y$,
有$\dfrac{1}{x} \leq \dfrac{1}{y}$,因此 $\dfrac{1}{k}-\dfrac{1}{y} \leq \dfrac{1}{y}$,
即$y \leq 2k$。这样,只需要在$2k$范围之类枚举$y$,然后根据$y$算出$x$即可。

\section{枚举排列} %%%%%%%%%%%%%%%%%%%%%%%%%%%%%%

\subsection{生成1~n的排列}

\subsection{生成可重复集合的排列}

\subsection{下一个排列}

\section{子集生成} %%%%%%%%%%%%%%%%%%%%%%%%%%%%%%

\subsection{增量构造法}

\subsection{位向量法}

\subsection{二进制法}
