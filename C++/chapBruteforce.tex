\chapter{暴力枚举法}
暴力枚举法(brute force enumeration)又称为暴力搜索法(Brute-force search),详细定义见Wikipedia, \myurl{http://en.wikipedia.org/wiki/Brute_force_search}

\section{枚举排列} %%%%%%%%%%%%%%%%%%%%%%%%%%%%%%
枚举排列,即输出某个集合的所有排列。例如,集合{1,2,3}的所有排列是{(1,2,3),(1,3,2),(2,1,3),(2,3,1),(3,1,2),(3,2,1)}。

\subsection{生成1到n的全排列}

\subsubsection{描述}
给定一个正整数n,输出1到n的所有排列。

\subsubsection{分析}
我们尝试用递归的思路:先输出以1开头的排列(这一步是递归调用),然后输出以2开头的排列(又是递归调用),...,最后才是以n开头的排列。

以1开头的排列,第一位是1,后面是2到9的排列,这是一个子问题,可以接着递归。

伪代码如下:
\begin{Code}
void print_permutation(序列P,集合S) {
    if (S为空) 输出序列P
    else {
        for(按照从小到大的顺序依次考虑S中的每个元素e) {
            print_permutation(在A的末尾添加e后得到的新序列, S-{e});
        }
    }
}
// 调用
print_permutation({}, S);
\end{Code}

\subsubsection{代码}
下面考虑用C语言实现。不难想到用数组表示P和S。由于P和S是互补的,它们二者知道其中给一个,另一个就完全确定了,因此不用保存P。

C语言实现如下:

\begin{Codex}[label=print_permutation_n.c]
#include <stdio.h>
#include <stdlib.h>

/*
 * @brief 输出1到n的全排列
 * @param[in] n n
 * @param[in] cur 当前进行到哪个位置
 * @param[out] 存放一个排列
 * @return 无
 */
static void print_permutation_r(int n, int cur, int P[]) {
    int i, j;
    if (cur == n) { // 收敛条件
        for (i = 0; i < n; i++)
            printf("%d", P[i]);
        printf("\n");
    }

    // 扩展状态,尝试在A[cur]中填各种整数i,按从小到大的顺序
    for (i = 1; i <= n; i++) {
        int used = 0;
        for (j = 0; j < cur; j++)
            if (P[j] == i)
                used = 1; // 如果i已经在A[0]~A[cur-1]出现过,则不能再选
        if (!used) {
            P[cur] = i;
            print_permutation_r(n, cur + 1, P); // 递归调用
            // 不需要恢复P[cur],返回上层时时会被覆盖
        }
    }
}

/**
 * @brief 输出1到n的全排列
 * @param[in] n n
 * @return 无
 */
void print_permutation(int n) {
    int *P = (int*)malloc(n * sizeof(int));
    print_permutation_r(n, 0, P);
    free(P);
    return;
}

//test
int main() {
    print_permutation(3);
    return 0;
}
\end{Codex}


\subsection{生成可重集的排列}

\subsubsection{描述}
如果把问题改成,给定一个数组S,按字典序输出所有排列。

\subsubsection{分析}
此时S不再仅限于整数,可以是字符等。

首先想到可以利用上一题的思路,先输出1到n的全排列,然后把打印语句改成\fn{printf("\%c", S[P[i]-1])}即可。

但是这个方法有点问题,当集合中有重复元素时,它不能正确处理。例如S="AAA",它会打印出6个AAA。见下面的代码。

\begin{Code}
#include <stdio.h>
#include <stdlib.h>

/*
 * @brief 输出1到n的全排列,把数字当做下标
 * @param[in] S 字符集合
 * @param[in] n n
 * @param[in] cur 当前进行到哪个位置
 * @param[out] 存放一个排列
 * @return 无
 */
static void print_permutation_r(char S[], int n, int cur, int P[]) {
    int i, j;
    if (cur == n) { // 递归边界
        for (i = 0; i < n; i++)
            printf("%c", S[P[i]-1]);
        printf("\n");
    } else {
        // 尝试在A[cur]中填各种整数i,按从小到大的顺序
        for (i = 1; i <= n; i++) {
            int ok = 1;
            for (j = 0; j < cur; j++)
                if (P[j] == i)
                    ok = 0; // 如果i已经在A[0]~A[cur-1]出现过,则不能再选
            if (ok) {
                P[cur] = i;
                print_permutation_r(S, n, cur + 1, P); // 递归调用
            }
        }
    }
}

/**
 * @brief 输出字符集合的全排列
 * @param[in] S 字符集合
 * @param[in] n n
 * @return 无
 */
void print_permutation(char S[], int n) {
    int *P = (int*)malloc(n * sizeof(int));
    print_permutation_r(S, n, 0, P);
    free(P);
    return;
}

//test
int main() {
    char *S="ABC";
    char *S1="AAA";
    print_permutation(S,3);
    print_permutation(S1,3); // 不能正确处理可重集
    return 0;
}
\end{Code}

再换一个思路,还是在上一节的代码上进行修改,把\fn{if(P[j]==i)}和\fn{P[cur]=i}分别改成\fn{if(P[j]==S[i])}和\fn{P[cur]=S[i]}。这样,只要把S中的所有元素按从小到大的顺序排序,然后调用\fn{print_permutation_r(S, n, 0, P)}即可。

这个方法看上去不错,可惜还是有个小问题。例如\fn{S="AAA"},程序什么也不输出(正确答案应该是唯一的全排列AAA),原因在于,我们禁止S数组中出现重复,而在S中本来就有重复元素,这个“禁令”是错误的。

解决方法是,统计P[0]到P[cur-1]中S[i]出现的次数c1,以及S数组中S[i]的出现次数c2,只要c1<c2,就能继续选择S[i]。

结果又如何呢?这次有输出了,可是输出了27个AAA。遗漏是没有了,可是出现了重复。程序先把第1个A作为开头,递归调用结束后用第2个A作为开头,递归调用结束后用第3个A作为开头。每次输出3个排列,共27个。

枚举时应该\textbf{不重不漏}地取遍集合的所有元素。由于数组已经排序,所以只需要检查当前元素和前一个元素不相同,就可以做到不重不漏了。即只需要在\fn{for (i = 1; i <= n; i++)}和其后的花括号之间加上\fn{if(i==0 \&\& S[i] != S[i-1])}。见下面的代码。

\subsubsection{代码}

\begin{Codex}[label=print_permutation.c]
#include <stdio.h>
#include <stdlib.h>

/*
 * @brief 输出1到n的全排列
 * @param[in] n n
 * @param[in] cur 当前进行到哪个位置
 * @param[out] 存放一个排列
 * @return 无
 */
static void print_permutation_r(int n, int cur, int P[]) {
    int i, j;
    if (cur == n) { // 收敛条件
        for (i = 0; i < n; i++)
            printf("%d", P[i]);
        printf("\n");
    }

    // 扩展状态,尝试在A[cur]中填各种整数i,按从小到大的顺序
    for (i = 1; i <= n; i++) {
        int used = 0;
        for (j = 0; j < cur; j++) {
            if (P[j] == i) {
                used = 1; // 如果i已经在A[0]~A[cur-1]出现过,则不能再选
                break;
            }
        }
        if (!used) {
            P[cur] = i;
            print_permutation_r(n, cur + 1, P); // 递归调用
            // 不需要恢复P[cur],返回上层时时会被覆盖
        }
    }
}

/**
 * @brief 输出1到n的全排列
 * @param[in] n n
 * @return 无
 */
void print_permutation(int n) {
    int *P = (int*)malloc(n * sizeof(int));
    print_permutation_r(n, 0, P);
    free(P);
    return;
}

//test
int main() {
    print_permutation(3);
    return 0;
}
\end{Codex}


\subsection{下一个排列}
还可以利用STL中的\fn{std::next_permutation()},或者自己实现,见第 \S \ref{sec:nextpermutation}节的\fn{next_permutation()}。

\subsubsection{代码}

\begin{Codex}[label=print_permutation_next.c]
#include <stdio.h>
#include <stdlib.h>
#include <algorithm>

/**
 * @brief 输出字符集合的全排列,利用std::next_permutation
 * @param[in] S 字符集合
 * @param[in] n n
 * @return 无
 */
void print_permutation(char S[], int n) {
    std::sort(&S[0], &S[n]);
    do {
        for(int i = 0; i < n; i++) printf("%c", S[i]);
        printf("\n");
    }while(std::next_permutation(&S[0], &S[n]));
    return;
}

/* 等价于复制粘贴,这里为了节约篇幅,使用include,在OJ上提交时请用复制粘贴 */
#include "next_permutation.c"

/**
 * @brief 输出字符集合的全排列,利用第15章的next_permutation
 * @param[in] S 字符集合
 * @param[in] n n
 * @return 无
 */
void print_permutation1(char S[], int n) {
    std::sort(&S[0], &S[n]);
    int *N = (int*)malloc(n * sizeof(int));
    for (int i = 0; i < n; ++i) N[i]=S[i]-'A';
    do {
        for(int i = 0; i < n; i++) printf("%c", S[N[i]]);
        printf("\n");
    }while(std::next_permutation(&N[0], &N[n]));
    return;
}

//test
int main() {
    char S[]="ABC";
    char S1[]="AAA";
    print_permutation(S,3);
    print_permutation(S1,3);
    printf("\n\n");
    print_permutation1(S,3);
    print_permutation1(S1,3);
    return 0;
}
\end{Codex}


\section{子集生成} %%%%%%%%%%%%%%%%%%%%%%%%%%%%%%
给定一个集合,输出它所有的子集。为了简单起见,本节讨论的集合中没有重复元素。


\subsection{增量构造法}
一次选出一个元素,放或者不放到集合中。

\subsubsection{代码}

\begin{Codex}[label=subset.c]
#include <stdio.h>
#include <stdlib.h>

/**
 * @brief 增量构造法
 * @param[in] S 输入集合
 * @param[in] n 集合大小
 * @param[inout] P 某个子集
 * @param[in] cur p的当前位置
 * @param[in] ed S的当前位置,前面的元素已经选过了
 * @return 无
 */
void print_subset1(int *S, int n, int *P, int cur, int ed) {
    int i, j;
    for (i = ed; i < n; i++) {
        // 选择 S[i]
        P[cur] = S[i];
        for (j = 0; j <= cur; j++) printf("%d ", P[j]);
        printf("\n");
        // 不选择 S[i]
        print_subset1(S, n, P, cur + 1, i + 1);
    }
}
\end{Codex}


\subsection{位向量法}
开一个位向量B,B[i]=1表示选择S[i], B[i]=0表示不选择。

\subsubsection{代码}

\begin{Codex}[label=subset.c]
/**
 * @brief 位向量法
 * @param[in] S 输入集合
 * @param[in] n 集合大小
 * @param[in] B 位向量
 * @param[in] cur B的当前位置
 * @return 无
 */
void print_subset2(int *S, int n, char *B, int cur) {
    int i;
    if (cur == n) {
        for (i = 0; i < n; i++) if (B[i]) printf("%d ", S[i]);
        printf("\n");
        return;
    }
    B[cur] = 1;
    print_subset2(S, n, B, cur + 1);
    B[cur] = 0;
    print_subset2(S, n, B, cur + 1);
}
\end{Codex}


\subsection{二进制法}
前提:集合的元素不超过int位数。用一个int整数表示位向量,第i位为1,则表示选择S[i],为0则不选择。例如S=\{A,B,C,D\},则0110=6表示子集\{B,C\}。

这种方法最巧妙。因为它不仅能生成子集,还能方便的表示集合的并、交、差等集合运算。设两个集合的位向量分别为B1和B2,则B1|B2, B1\&B2, B1\^{}B2分别对应集合的并、交、对称差。

\subsubsection{代码}

\begin{Codex}[label=subset.c]
/**
 * @brief 二进制法
 * @param[in] S 输入集合
 * @param[in] n 集合大小
 * @param[in] B 位向量
 * @param[in] cur B的当前位置
 * @return 无
 */
void print_subset3(int *S, int n) {
    int i, j;
    for (i = 1; i < (1 << n); i++) {
        for (j = 0; j < n; j++)
            if (i & (1 << j)) printf("%d ", S[j]);
        printf("\n");
    }
}

int main() {
    int n, i;

    while(scanf("%d",&n) > 0) {
        int *S = (int*)malloc(n * sizeof(int));
        int *P = (int*)malloc(n * sizeof(int));
        char *B = (char*)malloc(n * sizeof(char));

        for(i = 0; i < n; i++) scanf("%d",&S[i]);

        print_subset1(S, n, P, 0, 0); putchar('\n');
        print_subset2(S, n, B, 0); putchar('\n');
        print_subset3(S, n);

        free(S);
        free(P);
        free(B);
    }
    return 0;
}
\end{Codex}
