\chapter{查找}

\section{折半查找} %%%%%%%%%%%%%%%%%%%%%%%%%%%%%%

\begin{Codex}[label=binary_search.c]
/** 数组元素的类型 */
typedef int elem_t;
/**
  * @brief 有序顺序表的折半查找算法.
  *
  * @param[in] a 存放数据元素的数组,已排好序
  * @param[in] n 数组的元素个数
  * @param[in] x 要查找的元素
  * @return 如果找到 x,则返回其下标。 如果找
  * 不到 x 且 x 小于 array 中的一个或多个元素,则为一个负数,该负数是大
  * 于 x 的第一个元素的索引的按位求补。 如果找不到 x 且 x 大于 array 中的
  * 任何元素,则为一个负数,该负数是(最后一个元素的索引加 1)的按位求补。 
  */
int binary_search(const elem_t a[], const int n, const elem_t x) {
    int left = 0, right = n -1, mid;
    while(left <= right) {
        mid = left + (right - left) / 2;
        if(x > a[mid]) {
            left = mid + 1;
        } else if(x < a[mid]) {
            right = mid - 1;
        } else {
            return mid;
        }
    }
    return -(left+1);
}
\end{Codex}


\section{哈希表} %%%%%%%%%%%%%%%%%%%%%%%%%%%%%%


\subsection{原理和实现}
\label{sec:hash-set}
哈希表处理冲突有两种方式,开地址法(Open Addressing)和闭地址法(Closed Addressing)。

闭地址法也即拉链法(Chaining),每个哈希地址里不再是一个元素,而是链表的首地址。

开地址法有很多方案,有线性探测法(Linear Probing)、二次探测法(Quodratic Probing)和双散列法(Double Hashing)等。

下面是拉链法的C语言实现。

\subsubsection{代码}
\begin{Codex}[label=hash_set.c]
#include <stdio.h>
#include <string.h>
#include <stdlib.h>

#ifndef __cplusplus
typedef char bool;
#define false 0
#define true 1
#endif

/** 每个元素所带的数据 */
typedef int hash_set_elem_t;

/** 哈希表容量,一定要大于元素最大个数  */
#define HASH_SET_CAPACITY  100001
/** 哈希表取模的质数,也即哈希桶的个数,小于 HASH_SET_CAPACITY. */
#define PRIME  99997

/** 哈希集合. */
typedef struct hash_set_t {
    int head[PRIME];  /** 首节点下标 */

    struct {
        hash_set_elem_t elem;
        int next;
    } node[HASH_SET_CAPACITY];  /** 静态链表 */

    int size; /** 实际元素个数 */

    int (*hash)(const hash_set_elem_t *elem); /** 元素的哈希函数 */
    bool (*cmp)(const hash_set_elem_t *elem1,
            const hash_set_elem_t *elem2);  /** 元素的比较函数 */
} hash_set_t;

/** 创建一个哈希集合. */
hash_set_t* hash_set_create(int (*hash)(const hash_set_elem_t *elem),
        bool (*cmp)(const hash_set_elem_t *elem1,
                const hash_set_elem_t *elem2)) {
    hash_set_t *result = (hash_set_t*)malloc(sizeof(hash_set_t));
    memset(result->head, -1, sizeof(result->head));
    memset(result->node, -1, sizeof(result->node));
    result->size = 0;
    result->hash = hash;
    result->cmp = cmp;
    return result;
}

/** 销毁哈希集合. */
void hash_set_destroy(hash_set_t *set) {
    free(set);
    set = NULL;
}

/** 查找某个元素是否存在. */
bool hash_set_find(const hash_set_t *set, const hash_set_elem_t *elem) {
    int i;
    int hash = set->hash(elem);

    for (i = set->head[hash]; i != -1; i = set->node[i].next) {
        if (set->cmp(elem, &(set->node[i].elem))) {
            return true;
        }
    }
    return false;
}

/** 添加一个元素,如果已存在则添加失败. */
bool hash_set_add(hash_set_t *set, const hash_set_elem_t *elem) {
    int i;
    int hash = set->hash(elem);

    for (i = set->head[hash]; i != -1; i = set->node[i].next) {
        if (set->cmp(elem, &(set->node[i].elem))) {
            return false; /* 已经存在 */
        }
    }
    /* 不存在,则插入在首节点之前 */
    set->node[set->size].next = set->head[hash];
    set->node[set->size].elem = *elem;
    set->head[hash] = set->size++;
    return true;
}
\end{Codex}


\subsection{Babelfish}


\subsubsection{描述}
You have just moved from Waterloo to a big city. The people here speak an incomprehensible dialect of a foreign language. Fortunately, you have a dictionary to help you understand them.


\subsubsection{输入}
Input consists of up to 100,000 dictionary entries, followed by a blank line, followed by a message of up to 100,000 words. Each dictionary entry is a line containing an English word, followed by a space and a foreign language word. No foreign word appears more than once in the dictionary. The message is a sequence of words in the foreign language, one word on each line. Each word in the input is a sequence of at most 10 lowercase letters.


\subsubsection{输出}
Output is the message translated to English, one word per line. Foreign words not in the dictionary should be translated as "eh".


\subsubsection{样例输入}
\begin{Code}
dog ogday
cat atcay
pig igpay
froot ootfray
loops oopslay

atcay
ittenkay
oopslay
\end{Code}


\subsubsection{样例输出}
\begin{Code}
cat
eh
loops
\end{Code}


\subsubsection{分析}
最简单的方法是,把输入的单词对,存放在一个数组,排好序,查找的时候每次进行二分查找。

更快的方法是,用HashMap。C++有\fn{map},C++ 11以后有\fn{unordered_map},比\fn{map}快。也可以自己实现哈希表。


\subsubsection{代码}
使用 \fn{map} 。

\begin{Codex}[label=babelfish_map.cpp]
/* POJ 2503 Babelfish , http://poj.org/problem?id=2503 */
#include <cstdio>
#include <map>
#include <string>
#include <cstring>

using namespace std;

/** 字符串最大长度 */
const int MAX_WORD_LEN = 10;

int main() {
    char line[MAX_WORD_LEN * 2 + 1];
    char s1[MAX_WORD_LEN + 1], s2[MAX_WORD_LEN + 1];
    map<string, string> dict;

    while (gets(line) && line[0] != 0) {
        sscanf(line, "%s %s", s1, s2);
        dict[s2] = s1;
    }

    while (gets(line)) {
        if (dict[line].length() == 0) puts("eh");
        else printf("%s\n", dict[line].c_str());
    }
    return 0;
}
\end{Codex}


自己实现哈希表。

\begin{Codex}[label=babelfish.c]
/* POJ 2503 Babelfish , http://poj.org/problem?id=2503 */
/** 单词最大长度. */
#define MAX_WORD_LEN   11

/** 每个元素所带的数据 */
typedef struct hash_set_elem_t {
    char english[MAX_WORD_LEN];
    char foreign[MAX_WORD_LEN];
} hash_set_elem_t;

/** 哈希表容量,一定要大于元素最大个数  */
#define HASH_SET_CAPACITY  100001
/** 哈希表取模的质数,也即哈希桶的个数,小于 HASH_SET_CAPACITY. */
#define PRIME  99997

/* 等价于复制粘贴,这里为了节约篇幅,使用include,在OJ上提交时请用复制粘贴 */
#include "hash_set.c"  /* 见“查找->哈希表”这节 */

/* 与hash_set_find() 略有不同,用类似于HashMap的方式 */
bool hash_set_get(const hash_set_t *set, hash_set_elem_t *elem) {
    int i;
    int hash = set->hash(elem);

    for (i = set->head[hash]; i != -1; i = set->node[i].next) {
        if (set->cmp(elem, &(set->node[i].elem))) {
            // 返回对应 english, foreigh 作为key, english作为value
            strncpy(elem->english, set->node[i].elem.english, MAX_WORD_LEN-1);
            return true;
        }
    }
    return false;
}

int elem_hash(const hash_set_elem_t *elem) {
    const char *str = elem->foreign;
    unsigned int r = 0;
    int len = strlen(str);
    int i;

    for (i = 0; i < len; i++) {
        r = (r * 31 + str[i]) % PRIME;
    }
    return r % PRIME;
}

bool elem_cmp(const hash_set_elem_t *elem1, const hash_set_elem_t *elem2) {
    return strcmp(elem1->foreign, elem2->foreign) == 0;
}

int main() {
    char line[MAX_WORD_LEN * 2];
    hash_set_t *set = hash_set_create(elem_hash, elem_cmp);
    hash_set_elem_t e;

    while (gets(line) && line[0] != 0) {
        sscanf(line, "%s %s", e.english, e.foreign);
        hash_set_add(set, &e);
    }
    while (gets(e.foreign)) {
        if (hash_set_get(set, &e)) printf("%s\n", e.english);
        else printf("eh\n");
    }
    hash_set_destroy(set);
    return 0;
}
\end{Codex}


\subsubsection{相关的题目}
与本题相同的题目:
\begindot
\item POJ 2503 Babelfish, \myurl{http://poj.org/problem?id=2503}
\myenddot

与本题相似的题目:
\begindot
\item  无
\myenddot
