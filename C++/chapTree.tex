\chapter{树}

\section{二叉树的遍历} %%%%%%%%%%%%%%%%%%%%%%%%%%%%%%

\begin{Codex}[label=binary_tree.cpp]
#include <stack>
#include <queue>

 /*
  *@struct
  *@brief 二叉树结点
  */
typedef struct binary_tree_node_t {
    binary_tree_node_t *lchild;   /* 左孩子*/
    binary_tree_node_t *rchild;   /* 右孩子*/
    void* data; /* 结点的数据*/
}binary_tree_node_t;

/** 
  * @brief 先序遍历,递归.
  * @param[in] root 根结点
  * @param[in] visit 访问数据元素的函数指针
  * @return 无
  */
void pre_order_r(const binary_tree_node_t *root, 
                 int (*visit)(void*)) {
    if(root != NULL) {
        (void)visit(root->data);
        pre_order_r(root->lchild, visit);
        pre_order_r(root->rchild, visit);
    }
}

/** 
  * @brief 中序遍历,递归.
  */
void in_order_r(const binary_tree_node_t *root, 
                int (*visit)(void*)) {
    if(root != NULL) {
        pre_order_r(root->lchild, visit);
        (void)visit(root->data);
        pre_order_r(root->rchild, visit);
    }
}

/** 
  * @brief 后序遍历,递归.
  */
void post_order_r(const binary_tree_node_t *root, 
                  int (*visit)(void*)) {
    if(root != NULL) {
        pre_order_r(root->lchild, visit);
        pre_order_r(root->rchild, visit);
        (void)visit(root->data);
    }
}

/** 
 * @brief 先序遍历,非递归.
 */
void pre_order(const binary_tree_node_t *root, 
               int (*visit)(void*)) {
    const binary_tree_node_t *p;
    std::stack<const binary_tree_node_t *> s;

    p = root;

    if(p != NULL) {
        s.push(p);
    }

    while(!s.empty()) {
        p = s.top();
        s.pop();
        visit(p->data);
        if(p->rchild != NULL) {
            s.push(p->rchild);
        }
        if(p->lchild != NULL) {
            s.push(p->lchild);
        }
    }
}

/** 
 * @brief 中序遍历,非递归.
 */
void in_order(const binary_tree_node_t *root, 
              int (*visit)(void*)) {
    const binary_tree_node_t *p;
    std::stack<const binary_tree_node_t *> s;

    p = root;

    while(!s.empty() || p!=NULL) {
        if(p != NULL) {
            s.push(p);
            p = p->lchild;
        } else {
            p = s.top();
            s.pop();
            visit(p->data);
            p = p->rchild;
        }
    }
}

/** 
 * @brief 后序遍历,非递归.
 */
void post_order(const binary_tree_node_t *root, 
                int (*visit)(void*)) {
    /* p,正在访问的结点,q,刚刚访问过的结点*/
    const binary_tree_node_t *p, *q;
    std::stack<const binary_tree_node_t *> s;

    p = root;

    do {
        while(p != NULL) { /* 往左下走*/
            s.push(p);
            p = p->lchild;
        }
        q = NULL;
        while(!s.empty()) {
            p = s.top();
            s.pop();
            /* 右孩子不存在或已被访问,访问之*/
            if(p->rchild == q) {
                visit(p->data);
                q = p; /* 保存刚访问过的结点*/
            } else {
                /* 当前结点不能访问,需第二次进栈*/
                s.push(p);
                /* 先处理右子树*/
                p = p->rchild;
                break;
            }
        }
    }while(!s.empty());
}

/** 
 * @brief 层次遍历,也即BFS.
 *
 * 跟先序遍历一模一样,唯一的不同是栈换成了队列
 */
void level_order(const binary_tree_node_t *root, 
                 int (*visit)(void*)) {
    const binary_tree_node_t *p;
    std::queue<const binary_tree_node_t *> q;

    p = root;

    if(p != NULL) {
        q.push(p);
    }

    while(!q.empty()) {
        p = q.front();
        q.pop();
        visit(p->data);
        if(p->lchild != NULL) { /*先左后右或先右后左无所谓*/
            q.push(p->lchild);
        }
        if(p->rchild != NULL) {
            q.push(p->rchild);
        }
    }
}
\end{Codex}

\section{重建二叉树} %%%%%%%%%%%%%%%%%%%%%%%%%%%%%%
\begin{Codex}[label=binary_tree.cpp]
/** 
 * @brief 给定前序遍历和中序遍历,输出后序遍历.
 *
 * @param[in] n 序列的长度
 * @param[in] pre 前序遍历的序列
 * @param[in] in 中序遍历的序列
 * @param[out] post 后续遍历的序列
 * @return 无
 */
void build(const int n, const char * pre, const char *in, char *post) {
	if(n <= 0) return;
	int p = strchr(in, pre[0]) - in;
	build(p, pre + 1, in, post);
	build(n - p - 1, pre + p + 1, in + p + 1, post + p);
	post[n - 1] = pre[0];
}

// 测试
// BCAD CBAD,输出 CDAB
// DBACEGF ABCDEFG,输出 ACBFGED
int build_test() {
    const int MAX = 64;
    char pre[MAX] = {0};
    char in[MAX] = {0};
    char post[MAX] = {0};
    scanf("%s%s", pre, in);
    
    const int n = strlen(pre);
    build(n, pre, in, post);
    printf("%s\n", post);
}
\end{Codex}

\section{堆} %%%%%%%%%%%%%%%%%%%%%%%%%%%%%%

\subsection{堆的C语言实现}
C++可以直接使用\fn{std::priority_queue}。

\begin{Codex}[label=heap.c]
/** @file heap.c
 * @brief 堆,默认为小根堆,即堆顶为最小.
 * @author soulmachine@gmail.com
 * @date 2010-8-1
 * @version 1.0
 */
#include <stdlib.h>  /* for malloc() */
#include <string.h>  /* for memcpy() */

#ifndef __cplusplus 
typedef char bool;
#define false 0
#define true 1
#endif

typedef int heap_elem_t; // 元素的类型

/**
 * @struct
 * @brief 堆的结构体
 */
typedef struct heap_t {
    int     size;   /** 实际元素个数 */
    int     capacity; /** 容量,以元素为单位 */
    heap_elem_t  *elems;   /** 堆的数组 */
    int (*cmp)(const heap_elem_t*, const heap_elem_t*);   /** 元素的比较函数 */
}heap_t;

/** 
 * @brief 堆的初始化.
 * @param[out] h 堆对象的指针
 * @param[out] capacity 初始容量
 * @param[in] cmp cmp 比较函数,小于返回-1,等于返回0
 * 大于返回1,反过来则是大根堆
 * @return 成功返回0,失败返回错误码
 */
int heap_init(heap_t *h, const int capacity, 
              int (*cmp)(const heap_elem_t*, const heap_elem_t*)) {
    h->size = 0;
    h->capacity = capacity;
    h->elems = (heap_elem_t*)malloc(capacity * sizeof(heap_elem_t));
    h->cmp = cmp;
    
    return 0;
}

/** 
 * @brief 释放堆.
 * @param[inout] h 堆对象的指针
 * @return 成功返回0,失败返回错误码
 */
int heap_uninit(heap_t *h) {
    h->size = 0;
    h->capacity = 0;
    free(h->elems);
    h->elems = NULL;
    h->cmp = NULL;

    return 0;
}


/** 
 * @brief 判断堆是否为空.
 * @param[in] h 堆对象的指针
 * @return 是空,返回true,否则返回false
 */
bool heap_empty(const heap_t *h) {
    if(h != NULL) {
        return h->size == 0;
    } else {
        return false;
    }
}

/** 
 * @brief 获取元素个数.
 * @param[in] s 堆对象的指针
 * @return 元素个数
 */
int heap_size(const heap_t *h) {
    return h->size;
}

/*
 * @brief 小根堆的自上向下筛选算法.
 * @param[in] h 堆对象的指针
 * @param[in] start 开始结点
 * @return 无
 */
void heap_sift_down(const heap_t *h, const int start) {
    int i = start;
    int j;
    const heap_elem_t tmp = h->elems[start];
       
    for(j = 2 * i + 1; j < h->size; j = 2 * j + 1) {
        if(j < (h->size - 1) && 
            // h->elems[j] > h->elems[j + 1]
            h->cmp(&(h->elems[j]), &(h->elems[j + 1])) > 0) {
                j++; /* j 指向两子女中小者*/
        }
        // tmp <= h->data[j]
        if(h->cmp(&tmp, &(h->elems[j])) <= 0) {
            break;
        } else {
            h->elems[i] = h->elems[j];
            i = j;
        }
    }
    h->elems[i] = tmp;
}
 
/* 
 * @brief 小根堆的自下向上筛选算法.
 * @param[in] h 堆对象的指针
 * @param[in] start 开始结点
 * @return 无
 */
void heap_sift_up(const heap_t *h, const int start) {
    int j = start;
    int i= (j - 1) / 2;
    const heap_elem_t tmp = h->elems[start];
 
    while(j > 0) {
        // h->data[i] <= tmp
        if(h->cmp(&(h->elems[i]), &tmp) <= 0) {
            break;
        } else {
            h->elems[j] = h->elems[i];
            j = i;
            i = (i - 1) / 2;
        }
    }
    h->elems[j] = tmp;
}

/** 
 * @brief 添加一个元素.
 * @param[in] h 堆对象的指针
 * @param[in] x 要添加的元素
 * @return 无
 */
void heap_push(heap_t *h, const heap_elem_t x) {
    if(h->size == h->capacity) { /*已满,重新分配内存*/
        heap_elem_t* tmp = 
            (heap_elem_t*)realloc(h->elems, h->capacity * 2 * sizeof(heap_elem_t));
        h->elems = tmp;
        h->capacity *= 2;
    }
 
    h->elems[h->size] = x;
    h->size++;

    heap_sift_up(h, h->size - 1);
}

/** 
 * @brief 弹出堆顶元素.
 * @param[in] h 堆对象的指针
 * @return 无
 */
void heap_pop(heap_t *h) {
    h->elems[0] = h->elems[h->size - 1];
    h->size --;
    heap_sift_down(h, 0);
}

/** 
 * @brief 获取堆顶元素.
 * @param[in] h 堆对象的指针
 * @return 堆顶元素
 */
heap_elem_t heap_top(const heap_t *h) {
    return h->elems[0];
}
\end{Codex}