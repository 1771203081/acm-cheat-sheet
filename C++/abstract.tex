\subsubsection{内容简介}
这个手册包含了一些经典题目的范例代码,经过仔细编写,编码规范良好,适合在纸上默写。

一个人,成为一个好的作家之前,他需要背诵大量经典段落,写下很多练习的模仿作品。
类似的,成为一个好的程序员之前,也需要些大量的练习代码,反复模仿经典的代码。

这本手册的定位,比ACM 模板库的代码少,题型比ACM简单,对代码有一些讲解。ACM代码库
功能全,很多难度很高,且整本手册都是代码,没有讲解。本手册中的每一个题目,都至少在两
本纸质书中出现过。

全书的代码,使用 纯C + STL的风格。本书中的代码规范,跟在公司中的工程规范略有不同,
为了使代码短(方便迅速实现):

\begindot
\item 所有代码都是单一文件。这是因为一般OJ网站,提交代码的时候只有一个文本框,如果还是
按照标准做法,比如分为头文件.h和源代码.cpp,无法在网站上提交;

\item 喜欢在全局定义一个最大整数,例如MAX。一般的OJ题目,都会有数据规模的限制,所以定义
一个常量MAX表示这个规模,可以不用动态分配内存,让代码实现更简单;

\item 经常使用全局变量。比如用几个全局变量,定义某个递归函数需要的数据,减少递归函数的参数
个数,就减少了递归时栈内存的消耗,可以说这几个全局变量是这个递归函数的“环境”。

\item 不提倡防御式编程。不需要检查malloc()/new 返回的指针是否为NULL;不需要检查内部函数入口
参数的有效性;使用纯C基于对象编程时,调用对象的成员方法,不需要检查对象自身是否为NULL。
\myenddot

本手册假定读者已经学过《数据结构》\footnote{《数据结构》,严蔚敏等著,清华大学出版社,
\myurl{http://book.douban.com/subject/2024655/}},
《算法》\footnote{《Algorithms》,Robert Sedgewick, Addison-Wesley Professional, \myurl{http://book.douban.com/subject/4854123/}}
这两门课,熟练掌握C++或Java。

本手册是开源的,项目地址:\myurl{https://github.com/soulmachine/acm-cheatsheet}


\subsubsection{更新记录}
\begindot
\item[] 2013-04-17 v0.2
\item[] 2013-04-14 v0.1
\myenddot
