\subsubsection{内容简介}
本书的目标读者是准备去北美找工作的码农,也适用于在国内找工作的码农,以及刚接触ACM算法竞赛的新手。

本书包含了一些经典题目的范例代码,经过精心编写,编码规范良好,适合在纸上默写。

怎么样才算是经典的算法题?一般经典的题目都有约定俗成的名称,例如“八皇后问题”,“0-1背包问题”等,这些名字已经固定下来了,类似于一个“成语”,一般说出名字,大家就都知道题目意思了,不用再解释题目内容,这就是所谓的“经典”。同时,本书的每一个题目,都至少在两本纸质书中出现过。

这本书的定位,与ACM算法竞赛类书籍不同。全书的题目比ACM竞赛简单,没有高难度的题目,但每道题目,都有详细生动的解释,还给出了可以直接在OJ上AC的代码。同时,题目的范围不限于算法竞赛,还包括了一些面试中常碰到的工程类题目。

全书的代码,使用“纯C + STL”的风格。本书中的代码规范,跟在公司中的工程规范略有不同,为了使代码短(方便迅速实现):

\begindot
\item 所有代码都是单一文件。这是因为一般OJ网站,提交代码的时候只有一个文本框,如果还是
按照标准做法,比如分为头文件.h和源代码.cpp,无法在网站上提交;

\item 喜欢在全局定义一个最大整数,例如MAX。一般的OJ题目,都会有数据规模的限制,所以定义
一个常量MAX表示这个规模,可以不用动态分配内存,让代码实现更简单;

\item 经常使用全局变量。比如用几个全局变量,定义某个递归函数需要的数据,减少递归函数的参数
个数,就减少了递归时栈内存的消耗,可以说这几个全局变量是这个递归函数的“环境”。

\item 不提倡防御式编程。不需要检查malloc()/new 返回的指针是否为NULL;不需要检查内部函数入口
参数的有效性;使用纯C基于对象编程时,调用对象的成员方法,不需要检查对象自身是否为NULL。
\myenddot

本手册假定读者已经学过《数据结构》\footnote{《数据结构》,严蔚敏等著,清华大学出版社,
\myurl{http://book.douban.com/subject/2024655/}},
《算法》\footnote{《Algorithms》,Robert Sedgewick, Addison-Wesley Professional, \myurl{http://book.douban.com/subject/4854123/}}
这两门课,熟练掌握C++或Java。

本手册是开源的,项目地址:\myurl{https://github.com/soulmachine/acm-cheatsheet}


\subsubsection{更新记录}
\begindot
\item[] 2013-07-20 发布v1.0,完成了大部分重要章节
\item[] 2013-05-19 完成了一些重要章节
\item[] 2013-04-14 创建Repo,开始编写
\myenddot
