\chapter{编程技巧}
把较大的数组放在main函数外,作为全局变量,这样可以防止栈溢出,因为栈的大小是有限制的。

如果能够预估栈,队列的上限,则不要用\fn{stack, queue},使用数组来模拟,这样速度最快。

输入数据一般放在全局变量,且在运行过程中不要修改这些变量。

在判断两个浮点数a和b是否相等时,不要用\fn{a==b},应该判断二者之差的绝对值\fn{fabs(a-b)}是否小于某个阈值,例如\fn{1e-9}。

判断一个整数是否是为奇数,用\fn{x \% 2 != 0},不要用\fn{x \% 2 == 1},因为x可能是负数。

用\fn{char}的值作为数组下标(例如,统计字符串中每个字符出现的次数),要考虑到\fn{char}可能是负数。有的人考虑到了,先强制转型为\fn{unsigned int}再用作下标,这仍然是错的。正确的做法是,先强制转型为\fn{unsigned char},再用作下标。这涉及C++整型提升的规则,就不详述了。
